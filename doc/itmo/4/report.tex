\documentclass{article}

\usepackage[utf8]{inputenc}
\usepackage[russian]{babel}
\usepackage[a4paper, margin=1in]{geometry}
\usepackage{graphicx}
\usepackage{amsmath}
\usepackage{wrapfig}
\usepackage{multirow}
\usepackage{mathtools}
\usepackage{pgfplots}
\usepackage{pgfplotstable}
\usepackage{setspace}
\usepackage{changepage}
\usepackage{caption}
\usepackage{csquotes}
\usepackage{hyperref}
\usepackage{listings}

\pgfplotsset{compat=1.18}
\hypersetup{
  colorlinks = true,
  linkcolor  = blue,
  filecolor  = magenta,      
  urlcolor   = darkgray,
  pdftitle   = {
    math-tool-report-approx-smirnov-victor-p32131
  },
}

\definecolor{codegreen}{rgb}{0,0.6,0}
\definecolor{codegray}{rgb}{0.5,0.5,0.5}
\definecolor{codepurple}{rgb}{0.58,0,0.82}
\definecolor{backcolour}{rgb}{0.99,0.99,0.99}

\lstdefinestyle{codestyle}{
  backgroundcolor=\color{backcolour},   
  commentstyle=\color{codegreen},
  keywordstyle=\color{magenta},
  numberstyle=\tiny\color{codegray},
  stringstyle=\color{codepurple},
  basicstyle=\ttfamily\footnotesize,
  breakatwhitespace=false,         
  breaklines=true,                 
  captionpos=b,                    
  keepspaces=true,                 
  numbers=left,                    
  numbersep=5pt,                  
  showspaces=false,                
  showstringspaces=false,
  showtabs=false,                  
  tabsize=2
}

\lstset{style=codestyle}

\begin{document}

\begin{titlepage}
    \begin{center}
        \begin{spacing}{1.4}
            \large{Университет ИТМО} \\
            \large{Факультет программной инженерии и компьютерной техники} \\
        \end{spacing}
        \vfill
        \textbf{
            \huge{Вычислительная математика.} \\
            \huge{Лабораторная работа №4.} \\
            \huge{Аппроксимация функций методом наименьших квадратов} \\
        }
    \end{center}
    \vfill
    \begin{center}
        \begin{tabular}{r l}
            Группа:  & P32131                  \\
            Студент: & Смирнов Виктор Игоревич \\
            Вариант: & 17                      \\
        \end{tabular}
    \end{center}
    \vfill
    \begin{center}
        \begin{large}
            2023
        \end{large}
    \end{center}
\end{titlepage}

\section*{Ключевые слова}
Аппроксимация, Метод Наименьших квадратов, МНК.

\section{Цель работы}

Цель работы - найти функцию, являющуюся наилучшим приближением
заданной табличной функции по методу наименьших квадратов.
А также был реализован программный модуль, предоставляющий API
для аппроксимации вещественнозначных функций с одной переменной.

\section{Вычислительная задача}

Вычислительная задача была поставлена следующим образом:
дана функция $f(x) = \frac{2x}{x^4+17}$, необходимо аппроксимировать
ее линейной функцией и параболой на интервале $[0, 2]$, табулируя
функцию с шагом $0.2$.

Перед решением следует все же ввести читателя в курс дела и рассказать
ему про МНК, не так ли? Поэтому отправим заинтересованного читателя
ознакамливаться со \href{https://ru.wikipedia.org/wiki/%D0%9C%D0%B5%D1%82%D0%BE%D0%B4_%D0%BD%D0%B0%D0%B8%D0%BC%D0%B5%D0%BD%D1%8C%D1%88%D0%B8%D1%85_%D0%BA%D0%B2%D0%B0%D0%B4%D1%80%D0%B0%D1%82%D0%BE%D0%B2}{статьей в Википедии}.

На языке программирования C++ был реализован "Случай полиномиальной модели".

\lstinputlisting[
    language={C++},
    caption={Реализация метода наименьших квадратов на языке C++},
    linerange={13-56}
]{../../../src/Mathematica/Functional/Approx/LeastSqare.hpp}

С использованием данной функциональности напишем простенький
тест для получения результатов вычисления:


\lstinputlisting[
    language={C++},
    caption={Тест для решения вычислительной задачи},
    linerange={10-53}
]{../../../test/Mathematica/Functional/Approx/PlaygroundTest.cpp}

\begin{lstlisting}[
    language={Bash},
    caption={Результаты вывода программы для решения вычислительной задачи}
]
( 0.100000, 0.011765 )
( 0.300000, 0.035277 )
( 0.500000, 0.058608 )
( 0.700000, 0.081206 )
( 0.900000, 0.101948 )
( 1.100000, 0.119150 )
( 1.300000, 0.130942 )
( 1.500000, 0.135977 )
( 1.700000, 0.134111 )
( 1.900000, 0.126531 )
Line: (0.024521 + (0.069030 * $1))
Line: STD = 0.015459
Line: R2 = 0.868067
Parabola: ((-0.010176 + (0.172604 * $1)) + (-0.051787 * ($1 ^ 2)))
Parabola: STD = 0.003526
Parabola: R2 = 0.993138
\end{lstlisting}

Итак, мы получили прямую линейного тренда
$y = 0.069030 \cdot x + 0.024521$ и
параболу $y = -0.051787 \cdot x ^ 2 +
    0.172604 \cdot x + -0.010176$. Видно, что
парабола имеет меньшее среднеквадратичное
отклонение, что толкает нас сделать вывод
о том, что кдвадратическая зависимость
лучше приближает заданную функцию, в чем
нас окончательно убеждают показатели
$R^2$.

\begin{tikzpicture}[>=stealth]
    \begin{axis}[
            xmin=-1,xmax=3,
            ymin=-0.3,ymax=0.3,
            axis x line=middle,
            axis y line=middle,
            axis line style=<->,
            xlabel={$x$},
            ylabel={$y$},
        ]
        \addplot[no marks,blue,<->] 
            expression[domain=-1:3,samples=100]{
                (2*x) / (x*x*x*x + 17)
            }
            node[pos=0.65,anchor=south west]{
                $y=\frac{2x}{x^4+17}$
            };
        \addplot[no marks,red,<->] 
            expression[domain=-1:3,samples=100]{
                0.069030 * x + 0.024521
            }
            node[pos=0.85,anchor=south west]{line};
        \addplot[no marks,green,<->] 
            expression[domain=-1:3,samples=100]{
                ((-0.010176 + (0.172604 * x)) + (-0.051787 * (x ^ 2)))
            }
            node[pos=0.85,anchor=south west]{parabola};
    \end{axis}
\end{tikzpicture}

\section{Программная реализация задачи}

Далее представлю реализации различных алгоритмов 
аппроксимации функций на языке программирования C++.
Стоит заметить, что все они так или иначе реализованы через
методе наименьших квадратов.

\subsection{Нахождение линии линейного тренда}

\lstinputlisting[
    language={C++},
    caption={Нахождение линии линейного тренда},
    linerange={8-39}
]{../../../src/Mathematica/Functional/Approx/Linear.hpp}

\subsection{Нахождение линии логарифмического тренда}

\lstinputlisting[
    language={C++},
    caption={Нахождение линии логарифмического тренда},
    linerange={6-36}
]{../../../src/Mathematica/Functional/Approx/Logarithmic.hpp}

\subsection{Нахождение линии степенного тренда}

\lstinputlisting[
    language={C++},
    caption={Нахождение линии степенного тренда},
    linerange={9-38}
]{../../../src/Mathematica/Functional/Approx/Power.hpp}

\subsection{Нахождение линии экспоненциального тренда}

\lstinputlisting[
    language={C++},
    caption={Нахождение линии экспоненциального тренда},
    linerange={9-41}
]{../../../src/Mathematica/Functional/Approx/Exponential.hpp}

\subsection{Ядро моделей аппроксимации}

\lstinputlisting[
    language={C++},
    caption={Ядро моделей аппроксимации},
    linerange={9-34}
]{../../../src/Mathematica/Functional/Approx/Model/Core.hpp}

\subsection{Модель, выбирающая лучшую из данных}

\lstinputlisting[
    language={C++},
    caption={Модель, выбирающая лучшую из данных},
    linerange={10-46}
]{../../../src/Mathematica/Functional/Approx/Model/ChoiceOptimal.hpp}

\subsection{Модель, приблищая функцию кусочно}

\lstinputlisting[
    language={C++},
    caption={Модель, приблищая функцию кусочно},
    linerange={13-50}
]{../../../src/Mathematica/Functional/Approx/Model/Partition.hpp}

\section{Вывод}

Выполнив данную лабораторную работу я познакомился
с базовыми методами аппроксимации функций. По сути 
все рассмотренные методы являются формой метода
наименьших квадратов (МНК). МНК позволяет 
достаточно точно приближать функции разной природы,
кажется, что эффективность построенного полинома растет
пропорционально его степени. Также хорошо показывает
себя метод кусочной аппроксимации функции.

\begin{thebibliography}{9}

    \bibitem{calc-math-demidovich-maron}
    Б.П. Демидович, И.А. Марон Основы вычислительной математики:
    учебное пособие — 1966 год.

    \bibitem{itmo-lecture} 
    Лекции Татьяны Алексеевны Малышевой

\end{thebibliography}

\end{document}
