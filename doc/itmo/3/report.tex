\documentclass{article}

\usepackage[utf8]{inputenc}
\usepackage[russian]{babel}
\usepackage[a4paper, margin=1in]{geometry}
\usepackage{graphicx}
\usepackage{amsmath}
\usepackage{wrapfig}
\usepackage{multirow}
\usepackage{mathtools}
\usepackage{pgfplots}
\usepackage{pgfplotstable}
\usepackage{setspace}
\usepackage{changepage}
\usepackage{caption}
\usepackage{csquotes}
\usepackage{hyperref}
\usepackage{listings}

\pgfplotsset{compat=1.18}
\hypersetup{
  colorlinks = true,
  linkcolor  = blue,
  filecolor  = magenta,      
  urlcolor   = darkgray,
  pdftitle   = {
    math-tool-report-approx-smirnov-victor-p32131
  },
}

\definecolor{codegreen}{rgb}{0,0.6,0}
\definecolor{codegray}{rgb}{0.5,0.5,0.5}
\definecolor{codepurple}{rgb}{0.58,0,0.82}
\definecolor{backcolour}{rgb}{0.99,0.99,0.99}

\lstdefinestyle{codestyle}{
  backgroundcolor=\color{backcolour},   
  commentstyle=\color{codegreen},
  keywordstyle=\color{magenta},
  numberstyle=\tiny\color{codegray},
  stringstyle=\color{codepurple},
  basicstyle=\ttfamily\footnotesize,
  breakatwhitespace=false,         
  breaklines=true,                 
  captionpos=b,                    
  keepspaces=true,                 
  numbers=left,                    
  numbersep=5pt,                  
  showspaces=false,                
  showstringspaces=false,
  showtabs=false,                  
  tabsize=2
}

\lstset{style=codestyle}

\begin{document}

\begin{titlepage}
    \begin{center}
        \begin{spacing}{1.4}
            \large{Университет ИТМО} \\
            \large{Факультет программной инженерии и компьютерной техники} \\
        \end{spacing}
        \vfill
        \textbf{
            \huge{Вычислительная математика.} \\
            \huge{Лабораторная работа №3.} \\
            \huge{Численное интегрирование} \\
        }
    \end{center}
    \vfill
    \begin{center}
        \begin{tabular}{r l}
            Группа:  & P32131                  \\
            Студент: & Смирнов Виктор Игоревич \\
            Вариант: & 17                      \\
        \end{tabular}
    \end{center}
    \vfill
    \begin{center}
        \begin{large}
            2023
        \end{large}
    \end{center}
\end{titlepage}

\section*{Ключевые слова}
Численные методы, интегрирование.

\section{Цель работы}
Цель данной работы - найти приближенное значение определенного
интеграла с требуемой точностью различными численными методами,
реализовав численные методы на языке программирования.

\section{Вычислительная часть}
Необходимо вычислить интеграл функции
$y = 3 x ^ 3 - 4 x ^ 2 + 7 x - 17$ на интервале
[1, 2].

\subsection{Вычисление вручную}

\begin{equation}
    f(x) = 3 x ^ 3 - 4 x ^ 2 + 7 x - 17 \\
\end{equation}
\begin{equation}
    \int f(x) = 3 x ^ 3 - 4 x ^ 2 + 7 x - 17 = \frac{3x^4}{4}-\frac{4x^3}{3} + \frac{7x^2}{2} - 17x + const
\end{equation}
\begin{equation}
    F(x) = \frac{3x^4}{4}-\frac{4x^3}{3} + \frac{7x^2}{2} - 17x  \\
\end{equation}
\begin{equation}
    \int_{1}^{2}f(x) = F(2) - F(1) = -\frac{55}{12} = -4.583(3)  \\
\end{equation}

\subsection{Вычисление по формуле Ньютона-Котеса при n = 5}

\begin{lstlisting}[
    language={Bash},
    caption={Результаты вывода программы 1}
]
$ math-tool integrator cotes5 1 2 0.0001
6
Report
Taken method: Cotes n = 5
Scope: [1.000000, 2.000000]
Accuracy: 0.0001
Result: -4.58341
\end{lstlisting}

\begin{table}[h]
    \centering
    \caption{Трассировка метода Котеса (n = 5)}
    \begin{tabular}{|c|c|c|c|c|}
        \hline
        n  & size  & prev     & curr     & diff        \\ \hline
        1  & 8     & -4.92739 & -4.75271 & 0.174676    \\ \hline
        2  & 16    & -4.75271 & -4.66735 & 0.0853653   \\ \hline
        3  & 32    & -4.66735 & -4.62517 & 0.0421784   \\ \hline
        4  & 64    & -4.62517 & -4.60421 & 0.0209617   \\ \hline
        5  & 128   & -4.60421 & -4.59376 & 0.0104488   \\ \hline
        6  & 256   & -4.59376 & -4.58854 & 0.00521638  \\ \hline
        7  & 512   & -4.58854 & -4.58594 & 0.00260618  \\ \hline
        8  & 1024  & -4.58594 & -4.58464 & 0.00130259  \\ \hline
        9  & 2048  & -4.58464 & -4.58398 & 0.000651167 \\ \hline
        10 & 4096  & -4.58398 & -4.58366 & 0.000325552 \\ \hline
        11 & 8192  & -4.58366 & -4.5835  & 0.000162768 \\ \hline
        12 & 16384 & -4.5835  & -4.58341 & 8.13822e-05 \\ \hline
    \end{tabular}
\end{table}

\subsection{Вычисление методом средних прямоугольников}

\begin{lstlisting}[
    language={Bash},
    caption={Результаты вывода программы 2}
]
$ math-tool integrator rect-m 1 2 0.0001
6
Report
Taken method: Rectangle Middle
Scope: [1.000000, 2.000000]
Accuracy: 0.0001
Result: -4.58335
\end{lstlisting}

\begin{table}[h]
    \centering
    \caption{Трассировка метода Средних Прямоугольников}
    \begin{tabular}{|c|c|c|c|c|}
        \hline
        n & size & prev     & curr     & diff        \\ \hline
        1 & 8    & -4.63281 & -4.5957  & 0.0371094   \\ \hline
        2 & 16   & -4.5957  & -4.58643 & 0.00927734  \\ \hline
        3 & 32   & -4.58643 & -4.58411 & 0.00231934  \\ \hline
        4 & 64   & -4.58411 & -4.58353 & 0.000579834 \\ \hline
        5 & 128  & -4.58353 & -4.58338 & 0.000144958 \\ \hline
        6 & 256  & -4.58338 & -4.58335 & 3.62396e-05 \\ \hline
    \end{tabular}
\end{table}


\subsection{Вычисление по формуле Трапеций}

\begin{lstlisting}[
    language={Bash},
    caption={Результаты вывода программы 3}
]
$ math-tool integrator trapeze 1 2 0.0001
6
Report
Taken method: Trapeze
Scope: [1.000000, 2.000000]
Accuracy: 0.0001
Result: -4.58331
\end{lstlisting}

\begin{table}[h]
    \centering
    \caption{Трассировка метода Трапеций}
    \begin{tabular}{|c|c|c|c|c|}
        \hline
        n & size & prev     & curr     & diff        \\ \hline
        1 & 8    & -4.48438 & -4.55859 & 0.0742188   \\ \hline
        2 & 16   & -4.55859 & -4.57715 & 0.0185547   \\ \hline
        3 & 32   & -4.57715 & -4.58179 & 0.00463867  \\ \hline
        4 & 64   & -4.58179 & -4.58295 & 0.00115967  \\ \hline
        5 & 128  & -4.58295 & -4.58324 & 0.000289917 \\ \hline
        6 & 256  & -4.58324 & -4.58331 & 7.24792e-05 \\ \hline
    \end{tabular}
\end{table}


\subsection{Вычисление по формуле Симпсона}

\begin{lstlisting}[
    language={Bash},
    caption={Результаты вывода программы 4}
]
$ math-tool integrator simpson 1 2 0.0001
6
Report
Taken method: Simpson
Scope: [1.000000, 2.000000]
Accuracy: 0.0001
Result: -4.58333
\end{lstlisting}

\begin{table}[h]
    \centering
    \caption{Трассировка метода Симпсона}
    \begin{tabular}{|c|c|c|c|c|}
        \hline
        n & size & prev     & curr     & diff        \\ \hline
        1 & 8    & -4.58333 & -4.58333 & 1.77636e-15 \\ \hline
    \end{tabular}
\end{table}

\subsection{Функция с разрывом - расходящийся интеграл}

\begin{lstlisting}[
    language={Bash},
    caption={Результаты вывода программы при расходящемся интеграле 1}
]
$ math-tool integrator simpson -1 2 0.0001
7
Report
Taken method: Simpson
Scope: [-1.000000, 2.000000]
Accuracy: 0.0001
No result: Can't integrate given function, maybe it diverges in given interval
\end{lstlisting}

\section{Анализ полученных результатов}


Начнем со случая, когда функция терпит разрыв в
точке на интервале. Если несобственный интеграл
расходится, то его интеграл стремится к бесконечности,
поэтому при каждом новом приближении мы получаем все большее
и большее число в результате, поэтому я просто
ограничил количество интераций 20ю, и если это кол-во
превышено кидаю ошибку.

В целом, сразу видно, насолько хорош Симпсон, хотя
это на самом деле из-за того, что наша исходная функция
является полиномом 4ой степени и данный метод своими
приближениями очень точно подходит под данную модель.

Метод трапеции и прямоугольников работают примерно одинаково.
А вот формула Ньютона-Котеса при n = 5 расстроила - не
стоила она потраченных сил. А все почему? Потому что
интерполяционные многочлены подвели.

\section{Фрагменты программ}

\begin{lstlisting}[
    language={C++},
    caption={Программы для численного интегрирования}
]
template <typename T>
using TrivialMethod = std::function<T(Function<T>, Interval<T>)>;

template <typename T>
using Method = std::function<T(Function<T>, Partition<T>)>;

template <typename T>
using PointPeeker = std::function<T(Interval<T>)>;

template <typename T>
TrivialMethod<T> trivial0(PointPeeker<T> peek) noexcept {
  return [peek](auto f, auto interval) {
    return f(peek(interval)) * interval.length();
  };
}

template <typename T>
TrivialMethod<T> trivial1() noexcept {
  return [](auto f, auto interval) {
    auto left = f(interval.left());
    auto right = f(interval.right());
    return (left + right) / 2 * interval.length();
  };
}

template <typename T>
TrivialMethod<T> trivial2() noexcept {
  return [](auto f, auto interval) {
    auto a = interval.left();
    auto b = interval.right();
    return (b - a) / 6 * (
        + 1 * f(a) 
        + 4 * f((a + b) / 2) 
        + 1 * f(b)
    );
  };
}

template <typename T>
TrivialMethod<T> trivial5() noexcept {
  return [](auto f, auto interval) {
    const auto s = interval.left();
    const auto l = interval.length() / 6;
    return interval.length() / 288 * (                     
        + 19 * f(s + 0 * l)
        + 75 * f(s + 1 * l) 
        + 50 * f(s + 2 * l) 
        + 50 * f(s + 3 * l) 
        + 75 * f(s + 4 * l) 
        + 19 * f(s + 5 * l) 
    );
  };
}

template <typename T>
class Cotes {
public:
  explicit Cotes(TrivialMethod<T> trivial)
      : trivialAreaUnderGraph(trivial) {}

  T areaUnderGraph(Function<T> function, Partition<T> partition) noexcept {
    T sum = 0;
    for (auto interval : partition) {
      sum += trivialAreaUnderGraph(function, interval);
    }
    return sum;
  }

private:
  TrivialMethod<T> trivialAreaUnderGraph;
};

template <typename T>
class Approx {
public:
  explicit Approx(T epsilon, Cotes<T> cotes)
      : epsilon(epsilon), cotes(cotes) {}

  T integrate(Function<T> f, Interval<T> scope) {
    auto size = 4;

    auto prev = cotes.areaUnderGraph(f, Partition<T>(scope, size));
    for (Index i = 0; i < iterationsLimit; i++) {
      size *= 2;

      auto curr = cotes.areaUnderGraph(f, Partition<T>(scope, size));

      if (std::abs(prev - curr) < epsilon) {
        return curr;
      }

      prev = curr;
    };
    throw Error::ProcessDiverges(
        "Can't integrate given function, "
        "maybe it diverges in given interval"
    );
  }

private:
  Count iterationsLimit = 20;
  T epsilon;
  Cotes<T> cotes;
};
\end{lstlisting}

\section{Вывод}


Выполняя данную лабораторную работу я познакомился с
основными численными методами интегрирования.

Интегрирование - вычислительно сложная задача, то есть
для нее требуется много ресурсов. Для сильно скачущих
функциях требуются довольно мелкое разбиение из-за
чего колчество итераций сильно возрастает. Кроме того
мы увеличиваем количество этих итераций при каждой неудаче.

С расходящимися интегралами могут быть проблемы: мы можем
попытаться вычислить значение функции на интервале, на котором
она не определена и получить Undefined Behaviour, а если она
будет просто стремится к бесконечности на заданном интервале,
мы просто потратим много сил на вычисление и в конце-концов 
сдадимся.

\end{document}
