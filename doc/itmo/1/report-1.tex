\documentclass{article}

\usepackage[utf8]{inputenc}
\usepackage[russian]{babel}
\usepackage[a4paper, margin=1in]{geometry}
\usepackage{graphicx}
\usepackage{amsmath}
\usepackage{wrapfig}
\usepackage{multirow}
\usepackage{mathtools}
\usepackage{pgfplots}
\usepackage{pgfplotstable}
\usepackage{setspace}
\usepackage{changepage}
\usepackage{caption}
\usepackage{csquotes}
\usepackage{hyperref}
\usepackage{listings}

\pgfplotsset{compat=1.18}
\hypersetup{
  colorlinks = true,
  linkcolor  = blue,
  filecolor  = magenta,      
  urlcolor   = darkgray,
  pdftitle   = {
    math-tool-report-lineqsys-smirnov-victor-p32131
  },
}

\definecolor{codegreen}{rgb}{0,0.6,0}
\definecolor{codegray}{rgb}{0.5,0.5,0.5}
\definecolor{codepurple}{rgb}{0.58,0,0.82}
\definecolor{backcolour}{rgb}{0.99,0.99,0.99}

\lstdefinestyle{codestyle}{
  backgroundcolor=\color{backcolour},   
  commentstyle=\color{codegreen},
  keywordstyle=\color{magenta},
  numberstyle=\tiny\color{codegray},
  stringstyle=\color{codepurple},
  basicstyle=\ttfamily\footnotesize,
  breakatwhitespace=false,         
  breaklines=true,                 
  captionpos=b,                    
  keepspaces=true,                 
  numbers=left,                    
  numbersep=5pt,                  
  showspaces=false,                
  showstringspaces=false,
  showtabs=false,                  
  tabsize=2
}

\lstset{style=codestyle}

\begin{document}

\begin{titlepage}
    \begin{center}
        \begin{spacing}{1.4}
            \large{Университет ИТМО} \\
            \large{Факультет программной инженерии и компьютерной техники} \\
        \end{spacing}
        \vfill
        \textbf{
            \huge{Вычислительная математика.} \\
            \huge{Лабораторная работа №1.} \\
            \huge{Решение систем линейных уравнений} \\
        }
    \end{center}
    \vfill
    \begin{center}
        \begin{tabular}{r l}
            Группа:  & P32131                  \\
            Студент: & Смирнов Виктор Игоревич \\
        \end{tabular}
    \end{center}
    \vfill
    \begin{center}
        \begin{large}
            2023
        \end{large}
    \end{center}
\end{titlepage}

\section*{Ключевые слова}
Системы линейных алгебраических уравнений, СЛАУ, алгоритмы,
вычислительные методы.

\section{Цель работы}
Целью проделанной работы было прежде всего ознакомление с
вычислительными методами решения систем линейных алгебраических
уравнений. Задача состояла в том, чтобы реализовать необходимые
алгоритмы на языке программирования, а также разработать
программное приложение с пользовательским интерфейсом.

\section{Описание метода}
Был реализован метод главных элементов
(\cite{calc-math-demidovich-maron}, c. 281).

\section{Реализация алгоритма}

\lstinputlisting[
    language={C++},
    caption={Реализация метода главных элементов на языке C++},
    linerange={17-143}
]{../src/math/eq/lin/sys/solve/gauss/method.hpp}

\section{Примеры использования программного приложения}
\begin{lstlisting}[
    language={Bash},
    caption={Пример использования приложения 1}
]
$ cat a.txt
3
 6  -1 -1   11.33
-1   6 -1   32
-1  -1  6   42

$ math-tool lineqsys a.txt
lineqsys gauss method results report
det: 196
triangle matrix:
       6       -1       -1  |    11.33
       0    5.833   -1.167  |    33.89
       0        0      5.6  |    50.67
result: { 4.666, 7.619, 9.047 }
error:  { -2.861e-06, 3.815e-06, 3.815e-06 }
\end{lstlisting}

\begin{lstlisting}[
    language={Bash},
    caption={Пример использования приложения 2}
]
$ cat null.txt
5
0 0 0 0 0  0
0 0 0 0 0  0
0 0 0 0 0  0
0 0 0 0 0  0
0 0 0 0 0  0

$ math-tool lineqsys null.txt
error: matrix can't be solved by gauss
\end{lstlisting}

\section{Вывод}
В процессе выполнения данной лабораторной работы я 
ознакомился с некоторыми вычислительными методами для
решения систем линейных алгебраических уравнений.
Было очень интересно и увлекательно реализовывать их
на языке C++.

\begin{thebibliography}{9}

    \bibitem{calc-math-demidovich-maron}
    Б.П. Демидович, И.А. Марон Основы вычислительной математики:
    учебное пособие — 1966 год.

\end{thebibliography}

\end{document}
